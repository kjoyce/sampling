\documentclass{homework}
\usepackage{lipsum}
\usepackage{alltt}
\usepackage{cancel}
\usepackage{amsthm}
\usepackage{cleveref}
\usepackage{upgreek}
\usepackage{mathrsfs}
\usepackage{tikz}
\usepackage{units}
\usepackage{slashbox}
\newtheorem{lemma}{Lemma}

\DeclareMathOperator{\cov}{cov}

\title{Kevin Joyce}
\course{Stat 542 - Sampling - Homework 1}
\author{Kevin Joyce}
\docdate{\today}
\begin{document} 
\newcommand{\figref}[1]{\figurename~\ref{#1}}
\renewcommand{\bar}{\overline}
\renewcommand{\hat}{\widehat}
\renewcommand{\SS}{\mathcal S}
\newcommand{\HH}{\mathscr H}
\newcommand{\mom}{\widetilde}
\newcommand{\mle}{\widehat \Uptheta}
\newcommand{\eps}{\varepsilon}
\newcommand{\todist}{\stackrel{D}\longrightarrow}
\newcommand{\toprob}{\stackrel{p}\longrightarrow}
\newcommand{\TTheta}{\overline{\underline \Theta} }
\newcommand{\del}{\partial}
\newcommand{\approxsim}{\overset{\cdotp}{\underset{\cdotp}{\sim}}}
\newcommand{\RSS}{\ensuremath{\mathrm{RSS}}}
\newcommand{\MSE}{\ensuremath{\mathrm{MSE}}}
\newcommand{\SE}{\ensuremath{\mathrm{SE}}}
\newcommand{\TSS}{\ensuremath{\mathrm{TSS}}}
\newcommand{\Var}{\ensuremath{\mathrm{Var}}}
\newcommand{\SSReg}{\ensuremath{\mathrm{SSReg}}}
\renewcommand{\a}[1]{{\color{red} \it #1}}

\begin{longproblem}
Suppose you would like to take an SRS of size $n$ from a list of $N$ units, but do not know the population size $N$ in advance.  Consider the following procedure:
\begin{enumerate}[(a)]
  \item Set $S_0 = \{1,2,\dots,n\}$, so that the initial sample for consideration consists of the first $n$ units on the list.
  \item For $k=1,2,\dots,$ generate a random number $u_k$ between $0$ and $1$.  If $u_k > n/(n+k)$, then set $S_k$ equal to $S_{k-1}$.  If $u_k\le n/(n+k)$, then select one of the units in $S_{k-1}$ at random, and replace it by unit $(n+k)$ to form $S_k$.
\end{enumerate}
Show that $S_{N-n}$ from this procedure is an SRS of size $n$. \emph{Hint:} Use induction.
\end{longproblem}
\begin{solution}
   Fix the sample size at $n$, and let $k=N-n$ be the difference in the size
   between the population to the sample.
   For clarity, denote $S_k:\Omega \to \{1,2,\dots,(n+k)\}^n$ a random variable whose realizations are denoted $S_k(\omega) = s_k$.  
   %Denote $S_k$ to be the random sample of indices from $1,2,\dots,(n+k)$
   %obtained on the $k$th step of the algorithm and $s_k$ its realization.  
   We must show that $S_k$ takes on each subset of $\{1,2,\dots,N=n+k\}$ of size $n$ with equal probability.  We proceed by induction on $k$ from
   $k=0$ to $k=N-n$.

   In the base case, $k=0$ implies $N=n$ and $S_0$ takes on only $s_0 =
   \{1,\dots,n\}$ with probability 1, which is the only subset of $\{1,\dots,N=n+0\}$ of size $n$. Hence it is trivally a
   random sample of all such subsets.

  Now, assume the induction hypothesis, i.e.~$S_k$ produces an SRS for a population of size $n+k$ -- equivalently,
  $$
   P(S_k = s_k) = \left. 1 \middle/ \binom{n+k}{n} \right.
  $$
  for each subset $s_k$ of size $n$ from $\{1,2,\dots,(n+k)\}$.
  Upon the realization $S_{k+1}=s_{k+1}$, either the sample remains fixed with $S_{k+1} = s_k$, in which case
   $u_{k+1} \in \left(\frac
  n{n+k+1},1\right)$ with probability $\left( 1 - \frac{n}{n+k+1}\right)=\left(
  \frac{k+1}{n+k+1} \right)$, or $S_{k+1}$ takes on $s_k$ with a randomly selected element replaced with
  $(n+k+1)$, in which case $u_{k+1} \in \left(0,\frac n{n+k+1}\right]$
  with probability $\left( \frac{k+1}{n+k+1} \right)$. If we assume that the
  selection of $u_{k+1}$ is independent of the generation of the sample from
  the previous step, then in the first case
  \begin{align*}
    P(S_{k+1} = s_{k+1}) 
    &= P(S_k = s_k) P\left(U_{k+1} \in \left(\frac n{n+k+1},1\right) \right) \\
    &=\binom {n+k}{n}^{-1} \left( \frac{k+1}{n+k+1} \right)\\
    &=\binom {n+k+1}n^{-1}
  \end{align*}
  In the second case, denote $s_k =\{a_1,a_2,\dots,a_n\}$, then $s_{k+1} =
  s_k\setminus\{a_i\}\cup\{k+1+n\}$ where $a_i$ is the realization of the
  random selection in the algorithm.  For each event where
  $\{a_1,a_2,\dots,a_n\}\setminus a_i \subset S_k$, there are $(n+k) - (n-1) = k+1$
  possible choices that fix each $a_1,\dots,a_n$ except for $a_i$.  Since each
  of those events are disjoint and have equal probability, $P(S_{k+1} = s_{k+1}
  | u_k,a_i) = P(\{a_1,\dots,a_n\}\setminus a_i \in S_k) = (k+1)\binom{n+k}n^{-1}$.  Now, assuming the independence
  of realization of $u_k$ and $a_i$ and the previous selection, we have
  \begin{align*}
    P(S_{k+1} = s_{k+1}) &= \left((k+1)\binom{n+k}n^{-1}\right)\cdot P\left(U_{k+1} \in \left(0,\frac n{n+k+1}\right] \right) \cdot P\Big( A_i = a_i \Big)\\
    &= (k+1)\binom{n+k}n^{-1}\cdot \frac n{n+k+1} \cdot \left( \frac 1n\right)\\
    &= \binom{n+k}{n}^{-1}\frac{k+1}{n+k+1}\\
    &=\binom {n+k+1}n^{-1}.
  \end{align*} 

  Note that the collection of possible realizations $S_{k+1} = s_{k+1}$ contains all realizations $S_k=s_k$ which is each subset of size $n$ $\{1,\dots,n+k\}$ by the induction hypothesis.  Certainly, $s_{k+1}$ is a subset of size $n$ of $\{1,2,\dots,n+k+1\}$ as possibly only $(n+k+1)$ is added to $s_k$ in the $(k+1)$th step, and if $\{a_1,\dots,a_n\}$ is a subset of $\{1,\dots,n+k+1\}$, then if $(n+k+1) = a_i$ for some $a_i$, then $\{a_1,\dots,a_i=(n+k+1),\dots,a_n\}$ is a possible realization in step two, otherwise $(n+k+1)$ is not $a_i$ for any $i$, and the subset is realized in step 1.  Hence each subset of size $n$ from $\{1,2,\dots,k+n+1\}$ is realized.
\end{solution}
\end{document} 


